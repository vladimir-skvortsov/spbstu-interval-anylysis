\documentclass{article}
\usepackage[utf8]{inputenc}
\usepackage[english, russian]{babel}
\usepackage{graphicx}
\graphicspath{ {./graphics/} }
\usepackage[a4paper, total={6in, 8in}]{geometry}
\pagestyle{empty} %
\usepackage[pageanchor]{hyperref}
\usepackage{enumitem}
\usepackage{listings}
\usepackage{amssymb}
\usepackage{amsmath}
\usepackage{indentfirst}
\usepackage{tikz}

\newcommand{\kaucher}{
  \begin{tikzpicture}
    \draw (0, 0) circle (0.6ex);
    \draw (-0.6ex, 0) -- (0.6ex, 0);
  \end{tikzpicture}%
}

\begin{document}
  \begin{titlepage}
    \begin{center}
      Санкт-Петербургский политехнический университет \\Петра Великого
    \end{center}

    \begin{center}
      Физико-механический институт
    \end{center}

    \begin{center}
      Высшая школа прикладной математики и вычислительной\\ физики
    \end{center}

    \vspace{8em}

    \begin{center}
      \textbf{Отчет по лабораторной работе №4}\\
      \textbf{“Интервальный анализ”}
    \end{center}

    \vspace{\fill}

    \begin{flushright}
      \noindentВыполнили студент группы 5030102/10201:
      \hfill
      Скворцов Владимир Сергеевич \\
    \end{flushright}
    Преподаватель: \hfill Баженов Александр Николаевич

    \vspace{12em}

    \begin{center}
      Санкт-Петербург\\
      2024
    \end{center}
  \end{titlepage}

  \tableofcontents

  \newpage

  \section{Постановка задачи}

  Определить параметры линейной регрессии

  \begin{equation} \label{eq:islau}
    \mathbf{y} = \beta_0 + \beta_1 \mathbf{x},
  \end{equation}

  где \( \mathbf{x} \) входные данные, \( \mathbf{y} \) интервальные выходные
  данные, \( \beta_0 \), \( \beta_1 \) --- параметры линейной регрессии.

  Для калибровки измерителя, на вход подаётся набор постоянных
  напряжений

  \begin{equation}
    X = \{ x_i \}.
  \end{equation}

  Для надёжности, для каждого значения \( x \) проводится 100 измерений.

  Получается набор интервальных выборок

  \begin{equation}
    \mathbf{Y} = \{ \mathbf{y}_k \}_{k=1}^{100}.
  \end{equation}

  \( \text{rad} \mathbf{y} = \frac{1}{2^N} \) В, \( N = 14 \).

  Связь кодов данных и В:

  \[ V = \text{Code} / 16384 - 0.5. \]

  Сделать оценки значений \( \mathbf{Y} \) двумя способами:

  \begin{itemize}
    \item in: как интервал между первым и третьим квартилем
    \item ex: как границы бокс-плота
  \end{itemize}

  Решить ИСЛАУ \ref{eq:islau} для внутренних и внешних оценок
  \( \mathbf{y} \)
  Построить множество решений \( \beta_0 \), \( \beta_1 \).
  Построить коридор совместных зависимостей.

  \section{Необходимая теория}

  \subsection{Интервальная мода}

  Пусть имеется интервальная выборка

  \[
    \mathbf{X} = \{ \mathbf{x}_i \}.
  \]

  Сформируем массив интервалов \( \mathbf{z} \) из концов интервалов
  \( \mathbf{X} \).

  Для каждого интервала \( \mathbf{z}_i \) подсчитываем число \( \mu_i \)
  интервалов из выборки \( \mathbf{X}_i \), включающих \( \mathbf{z}_i \).
  Максимальные \( \mu_i = \max \mu \) достигаются для индексного множества
  \( K \). Тогда можно найти интервальную моду как мультиинтервал

  \begin{equation}
    \text{mode} \mathbf{X} = \bigcup_{k \in K} \mathbf{z}_k.
  \end{equation}

  \section{Реализаця}

  Лабораторная работа выполнена на языке программирования Python. В ходе
  работы были также использованы библиотеки \verb!numpy! и
  \verb!matplotlib!.

  Ссылка на GitHub репозиторий:
  \href{https://github.com/vladimir-skvortsov/spbstu-interval-anylysis}
  {https://github.com/vladimir-skvortsov/spbstu-interval-anylysis}

  \section{Результаты}



  \section{Выводы}


\end{document}
