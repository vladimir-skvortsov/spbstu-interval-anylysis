\documentclass{article}
\usepackage[utf8]{inputenc}
\usepackage[english, russian]{babel}
\usepackage{graphicx}
\graphicspath{ {./graphics/} }
\usepackage[a4paper, total={6in, 8in}]{geometry}
\pagestyle{empty} %
\usepackage[pageanchor]{hyperref}
\usepackage{enumitem}
\usepackage{listings}
\usepackage{amssymb}
\usepackage{amsmath}
\usepackage{indentfirst}
\usepackage{tikz}

\newcommand{\kaucher}{
  \begin{tikzpicture}
    \draw (0, 0) circle (0.6ex);
    \draw (-0.6ex, 0) -- (0.6ex, 0);
  \end{tikzpicture}%
}

\begin{document}
  \begin{titlepage}
    \begin{center}
      Санкт-Петербургский политехнический университет \\Петра Великого
    \end{center}

    \begin{center}
      Физико-механический институт
    \end{center}

    \begin{center}
      Высшая школа прикладной математики и вычислительной\\ физики
    \end{center}

    \vspace{8em}

    \begin{center}
      \textbf{Отчет по лабораторной работе №3}\\
      \textbf{“Интервальный анализ”}
    \end{center}

    \vspace{\fill}

    \begin{flushright}
      \noindentВыполнили студент группы 5030102/10201:
      \hfill
      Скворцов Владимир Сергеевич \\
    \end{flushright}
    Преподаватель: \hfill Баженов Александр Николаевич

    \vspace{12em}

    \begin{center}
      Санкт-Петербург\\
      2024
    \end{center}
  \end{titlepage}

  \tableofcontents

  \newpage

  \section{Постановка задачи}

  Даны 2 интервальных выборки

  \begin{equation}
    \mathbf{X} = \{ \mathbf{x}_i \},
  \end{equation}
  \begin{equation}
    \mathbf{Y} = \{ \mathbf{y}_i \}.
  \end{equation}

  Взять \( \mathbf{X}, \mathbf{Y} \) из файлов данных, задав
  \( \text{rad} \mathbf{x} = \text{rad} \mathbf{y} = \frac{1}{2^N} \text{В} \),
  \( N = 14 \).

  Файлы данных:
  \begin{itemize}
    \item \emph{-0.205\_lvl\_side\_a\_fast\_data.bin}
    \item \emph{0.225\_lvl\_side\_a\_fast\_data.bin}
  \end{itemize}

  Связь кодов данных и В:

  \begin{equation*}
    V = N  / 16384 - 0.5
  \end{equation*}

  Сделать оценки констант \( a \), \( t \) в уравнениях:
  \begin{equation}
    \mathbf{X} + a = \mathbf{Y},
  \end{equation}
  \begin{equation}
    t\mathbf{X} = \mathbf{Y},
  \end{equation}

  Метод решения:

  \begin{equation}
    \hat a = \text{argmax} F(a, \mathbf{X}, \mathbf{Y}),
  \end{equation}

  где \( F \) --- функционал.

  В качестве функционала взять варианты:

  \begin{equation}
    \text{Ji} (a, \mathbf{X}, \mathbf{Y}),
  \end{equation}
  \begin{equation}
    \text{Ji} (a, \text{mode} \mathbf{X}, \text{mode} \mathbf{Y}),
  \end{equation}
  \begin{equation}
    \text{Ji} (a, \text{med}_K \mathbf{X}, \text{med}_K \mathbf{Y}),
  \end{equation}
  \begin{equation}
    \text{Ji} (a, \text{med}_P \mathbf{X}, \text{med}_P \mathbf{Y}),
  \end{equation}

  где \( \text{Ji} \) --- коэффициент Жаккара,
  \( \text{mode} \) --- интервальная мода,
  \( \text{med}_K \), \( \text{med}_P \) --- интервальные медианы Крейновича
  и Пролубникова.

  Сделать точечные и интервальные оценки, задавшись уровнем \( \alpha \).

  \section{Необходимая теория}

  \subsection{Интервальная мода}

  Пусть имеется интеральная выборка

  \[
    \mathbf{X} = \{ \mathbf{x}_i \}.
  \]

  Сформируем массив интервалов \( \mathbf{z} \) из концов интервалов
  \( \mathbf{X} \).

  Для каждого интервала \( \mathbf{z}_i \) подсчитываем число \( \mu_i \)
  интервалов из выборки \( \mathbf{X}_i \), включающих \( \mathbf{z}_i \).
  Максимальные \( \mu_i = \max \mu \) достигаются для индексного множества
  \( K \). Тогда можно найти интервальную моду как мультиинтервал

  \begin{equation}
    \text{mode} \mathbf{X} = \bigcup_{k \in K} \mathbf{z}_k.
  \end{equation}

  \section{Реализаця}

  Лабораторная работа выполнена на языке программирования Python. В ходе
  работы были также использованы библиотеки \verb!numpy! и
  \verb!matplotlib!.

  Ссылка на GitHub репозиторий:
  \href{https://github.com/vladimir-skvortsov/spbstu-interval-anylysis}
  {https://github.com/vladimir-skvortsov/spbstu-interval-anylysis}

  \clearpage

  \section{Результаты}

  \section{Выводы}
\end{document}
